% Options for packages loaded elsewhere
\PassOptionsToPackage{unicode}{hyperref}
\PassOptionsToPackage{hyphens}{url}
\PassOptionsToPackage{dvipsnames,svgnames,x11names}{xcolor}
%
\documentclass[
  letterpaper,
  DIV=11,
  numbers=noendperiod]{scrartcl}

\usepackage{amsmath,amssymb}
\usepackage{iftex}
\ifPDFTeX
  \usepackage[T1]{fontenc}
  \usepackage[utf8]{inputenc}
  \usepackage{textcomp} % provide euro and other symbols
\else % if luatex or xetex
  \usepackage{unicode-math}
  \defaultfontfeatures{Scale=MatchLowercase}
  \defaultfontfeatures[\rmfamily]{Ligatures=TeX,Scale=1}
\fi
\usepackage{lmodern}
\ifPDFTeX\else  
    % xetex/luatex font selection
\fi
% Use upquote if available, for straight quotes in verbatim environments
\IfFileExists{upquote.sty}{\usepackage{upquote}}{}
\IfFileExists{microtype.sty}{% use microtype if available
  \usepackage[]{microtype}
  \UseMicrotypeSet[protrusion]{basicmath} % disable protrusion for tt fonts
}{}
\makeatletter
\@ifundefined{KOMAClassName}{% if non-KOMA class
  \IfFileExists{parskip.sty}{%
    \usepackage{parskip}
  }{% else
    \setlength{\parindent}{0pt}
    \setlength{\parskip}{6pt plus 2pt minus 1pt}}
}{% if KOMA class
  \KOMAoptions{parskip=half}}
\makeatother
\usepackage{xcolor}
\setlength{\emergencystretch}{3em} % prevent overfull lines
\setcounter{secnumdepth}{5}
% Make \paragraph and \subparagraph free-standing
\ifx\paragraph\undefined\else
  \let\oldparagraph\paragraph
  \renewcommand{\paragraph}[1]{\oldparagraph{#1}\mbox{}}
\fi
\ifx\subparagraph\undefined\else
  \let\oldsubparagraph\subparagraph
  \renewcommand{\subparagraph}[1]{\oldsubparagraph{#1}\mbox{}}
\fi


\providecommand{\tightlist}{%
  \setlength{\itemsep}{0pt}\setlength{\parskip}{0pt}}\usepackage{longtable,booktabs,array}
\usepackage{calc} % for calculating minipage widths
% Correct order of tables after \paragraph or \subparagraph
\usepackage{etoolbox}
\makeatletter
\patchcmd\longtable{\par}{\if@noskipsec\mbox{}\fi\par}{}{}
\makeatother
% Allow footnotes in longtable head/foot
\IfFileExists{footnotehyper.sty}{\usepackage{footnotehyper}}{\usepackage{footnote}}
\makesavenoteenv{longtable}
\usepackage{graphicx}
\makeatletter
\def\maxwidth{\ifdim\Gin@nat@width>\linewidth\linewidth\else\Gin@nat@width\fi}
\def\maxheight{\ifdim\Gin@nat@height>\textheight\textheight\else\Gin@nat@height\fi}
\makeatother
% Scale images if necessary, so that they will not overflow the page
% margins by default, and it is still possible to overwrite the defaults
% using explicit options in \includegraphics[width, height, ...]{}
\setkeys{Gin}{width=\maxwidth,height=\maxheight,keepaspectratio}
% Set default figure placement to htbp
\makeatletter
\def\fps@figure{htbp}
\makeatother

\KOMAoption{captions}{tableheading}
\makeatletter
\@ifpackageloaded{caption}{}{\usepackage{caption}}
\AtBeginDocument{%
\ifdefined\contentsname
  \renewcommand*\contentsname{Table of contents}
\else
  \newcommand\contentsname{Table of contents}
\fi
\ifdefined\listfigurename
  \renewcommand*\listfigurename{List of Figures}
\else
  \newcommand\listfigurename{List of Figures}
\fi
\ifdefined\listtablename
  \renewcommand*\listtablename{List of Tables}
\else
  \newcommand\listtablename{List of Tables}
\fi
\ifdefined\figurename
  \renewcommand*\figurename{Figure}
\else
  \newcommand\figurename{Figure}
\fi
\ifdefined\tablename
  \renewcommand*\tablename{Table}
\else
  \newcommand\tablename{Table}
\fi
}
\@ifpackageloaded{float}{}{\usepackage{float}}
\floatstyle{ruled}
\@ifundefined{c@chapter}{\newfloat{codelisting}{h}{lop}}{\newfloat{codelisting}{h}{lop}[chapter]}
\floatname{codelisting}{Listing}
\newcommand*\listoflistings{\listof{codelisting}{List of Listings}}
\makeatother
\makeatletter
\makeatother
\makeatletter
\@ifpackageloaded{caption}{}{\usepackage{caption}}
\@ifpackageloaded{subcaption}{}{\usepackage{subcaption}}
\makeatother
\ifLuaTeX
  \usepackage{selnolig}  % disable illegal ligatures
\fi
\usepackage{bookmark}

\IfFileExists{xurl.sty}{\usepackage{xurl}}{} % add URL line breaks if available
\urlstyle{same} % disable monospaced font for URLs
\hypersetup{
  pdftitle={Syllabus},
  colorlinks=true,
  linkcolor={blue},
  filecolor={Maroon},
  citecolor={Blue},
  urlcolor={Blue},
  pdfcreator={LaTeX via pandoc}}

\title{Syllabus}
\author{}
\date{}

\begin{document}
\maketitle

\renewcommand*\contentsname{Table of contents}
{
\hypersetup{linkcolor=}
\setcounter{tocdepth}{3}
\tableofcontents
}
\section{Classroom expectations}\label{classexpectations}

\subsection{What you can expect from me}

\begin{itemize}
\tightlist
\item
  I will stay home if I am feeling sick and make arrangements to deliver
  the course material
\item
  I will work with you to arrange accommodations when you need them
\item
  I will respect your time by starting and ending class on time
\item
  I will answer your questions thoughtfully, and if I don't know the
  answer, I will follow up in a timely manner
\item
  I will embrace who you are as whole people
\item
  I will model respect, openness, and engagement, and foster a
  supportive and inclusive environment
\item
  I will be honest when I make mistakes, because failure is part of
  growing
\end{itemize}

\subsection{What I expect from you}

\begin{itemize}
\tightlist
\item
  That you will stay home if you are sick and contact me via email to
  arrange accommodations
\item
  That you genuinely attempt to engage with the course
\item
  That you ask questions if you are confused (you may do this privately
  -- there is no obligation to ask during class hours)
\item
  That you communicate with me when you have problems that interfere
  with your ability to engage with the coursework
\item
  That you treat your peers with respect and openness, and that you
  participate in creating an inclusive, supportive, and engaged
  classroom
\end{itemize}

\subsection{What is not expected}

\begin{itemize}
\tightlist
\item
  Perfection. Ever. It's a myth.
\item
  That you will `sit still' or ask for permission to leave the classroom
  to go to the bathroom or if you just need a minute.
\item
  That everyone will learn in the same way. You do not have to match
  some ``model student'' to do well in this class
\end{itemize}

\section{Assignments and Grading}\label{assignments}

Assignments fall into ``bundles,'' which contribute to your grade
equally. Your performance on each bundle determines your base grade (by
averaging the grade in each bundle using the 4-point GPA scale). Beyond
that, you can achieve grade boosts, which round your grade up, e.g.~from
a B to a B+, or a B+ to an A-.

You can learn more about each bundle below:

\hyperref[practice]{Practice}

\hyperref[groupwork]{Group Work}

\hyperref[quizzes]{Quizzes}

\hyperref[excellence]{Excellence}

\subsection{Grading Scheme}\label{gradingscheme}

At the end of the term, you will earn a letter grade for this course.
That letter grade will be determined following the Four-Square grading
chart (below). Each square represents a grading bundle, and you can earn
a score of 0-4 for each bundle. At the end of the term, I will calculate
an average score for you using the following equation:

Overall Score = (Practice score + Group Work score + Quizzes score +
Excellence score) / 4

\includegraphics[width=10.41667in,height=\textheight]{images/PHYS301foursquare.png}

The resulting overall score will correspond to a letter grade using the
\href{https://catalog.bates.edu/academics/academic-policies/grading-system/}{Bates
College 4-point GPA table}. You can then get up to three grade boosts,
each of which rounds your grade up (e.g from a B to B+ or from an A- to
A). The \hyperref[boosts]{grade boosts} are described more below.

If, for example, you achieved a 2 in Practice, a 4 in Group Work, a 3 in
Quizzes, and got a total of 11 ``Es'', your base grade would be (2.0 +
4.0 + 3.0 + 2.0)/4 = 2.75, which would be a B-. With round-ups, your
grade could go up to a B, B+, or A-.

\subsection{Grading scales}\label{grading-scales}

\subsubsection{Points}

Homeworks and pre-class quizzes will be graded on a points scale. Your
total grade for the bundle will be your average points on the
assignments. Remember that these are not graded on ``correctness'' but
rather on engaging in good faith.

\subsubsection{Completion}

The five final project components will each have objectives you must
meet for them to be considered ``complete.'' These will be clearly laid
out for each component so you know exactly what to do to complete them.

\subsubsection{E/M/R/N}

The unit tests and some components of the final project will be graded
on the following scale:

\begin{itemize}
\tightlist
\item
  E: excellent -- this is a thorough and correct response that
  demonstrates excellent understanding of the concepts and makes proper
  use of the mathematical skills expected in this class.
\item
  M: meets expectations -- this is a response that demonstrates solid
  understanding of the concepts but perhaps includes some small
  mathematical errors or minor conceptual mistakes.
\item
  R: unsatisfactory / does not meet expectations -- this is an answer
  that applies the concepts incorrectly, misunderstands the point of the
  question, does not complete the question, fails to follow
  instructions, and/or contains significant mathematical errors
\item
  N: no submission -- if you do not take the test, you will receive an
  N.
\end{itemize}

E and M are considered passing grades. U and N are considered failing
grades, but you have one chance to make up that grade, as every test
will be given twice.

\subsection{Grading categories}\label{categories}

\subsubsection{Pre-Class Quizzes}

In order to get the most out of class, you need to prepare. We will
spend most of our time in class practicing problem solving, and before
class you will need to read the appropriate sections of the book and
take a ``quiz''. Some readings are longer than others (and I've included
the page count in the
\href{https://docs.google.com/spreadsheets/d/1MBB7MfZfe-Gi5TbngnX-HuC6R6krND6FmvPM4rGqc0o/edit?usp=sharing}{course
schedule} so you can be prepared), but every reading quiz is 5
questions, some of which will be practice problems.

Given the purpose of the pre-class quizzes, I will not offer extensions,
except in case of a major medical or family emergency that causes you to
miss class.

\subsubsection{WHW and self-assessment}

There are seven homework sets (one every week except on test weeks) as
well as a self-reflection form.

You will turn in your work on the homework by scanning and uploading as
a PDF to Lyceum. These are due by class time on Friday. You will then
have a self-reflection due 72 hours later (by class time on Monday).

The problem sets are self-graded. Here's how that works:

\begin{itemize}
\tightlist
\item
  You get 50\% of the points just for submitting the homework on time.
\item
  When you submit your work, you will receive access to the (detailed)
  solutions.
\item
  The rest of the points come from comparing your work to the solutions
  and filling out a guided self-reflection form.
\item
  These forms are not intended to make you do ``extra work'' -- these
  are to help both you and me understand where you need more practice
  and more support.
\end{itemize}

You may request extensions on the homeworks using the
\href{https://forms.gle/eFx7y7FoSdoukKGC6}{extension form}.

\subsubsection{Unit tests}

There will be three tests on the material in this class. Each test will
be offered two times, two weeks apart, giving you two opportunities to
achieve your desired grade on the tests. The tests are graded on an
\hyperref[emun]{E/M/U/N scale}, and an E and M are both considered
passing grades.

You will be provided with an equation sheet for each test, which will
have some useful equations, integrals, trig identities, etc. You may
also bring a single-sided 8.5''x11'' sheet of paper with equations and
notes - hand-written (please speak to me if you need to type the sheet
and we will come up with a plan). This paper will be turned in with your
test.

If you have to miss class or leave early on a Friday due to an approved
excuse, you may take the test at the testing center. You must do this no
more than 2 days before the test is given and no more than 5 days after
(i.e.~starting the Wednesday before the test and ending the Wednesday
after the test), and you can request this option through AESS using this
\href{https://www.bates.edu/accessible-education-student-support/makeup-exam-requests/}{form}.
This is so I can provide timely feedback to everyone on the test. If you
are not sure your absence is excused or you feel you have a special
circumstance, come and talk to me about it and we will discuss the
options.

You don't have to do anything special to take the test a second time,
but I recommend coming to my office hours or scheduling a meeting with
me if you are at all confused about what you missed. I am more than
happy to help talk you through the test.

\subsubsection{Final Project}

By the end of this class, you will have had two semesters of instruction
in quantum mechanics and be well-versed in the fundamentals and how to
use the mathematics to derive real-world measurements. Now, you get a
chance to pick something about quantum mechanics that you want to know
more about and dig deeply.

This project is designed to be open-ended. I want you be as creative as
you want to be. If you want to con- nect quantum mechanics to the arts,
be my guest! If you want to do a deep dive into the historical
development of the ideas of quantum mechanics, please enlighten us! If
you want to do a computational project where you apply your programming
skills to a concept you learned in this class, that would be awesome! If
you want to teach a topic from the book that we haven't covered yet to
the rest of the class, we would love that! Want to record a podcast
interviewing scientists in the field? Great!

Think about where you want to stretch your understanding, and let that
guide you.

Since there is so much freedom, I'm building a lot of scaffolding into
the process, and that scaffolding will be the basis of the grade. So if
you want to take a big risk and try something new, you don't have to
worry that if it doesn't go well, you will fail. The scaffolding is
detailed on Lyceum, and there are five parts: the pitch, the proposal,
the annotated bibliography, the progress report, and the final
presentation and write-up. You must do the proposal and the final
presentation and write-up to pass the class -- completing other
components of the final project will contribute to higher base letter
grades (see the \hyperref[gradingscheme]{grading scheme} for more
details.)

\subsection{Grading boosts}\label{grading-boosts}

Your base grade will be calculated from the table above, but you can
round your grade up in a number of ways.

I will apply as many grade boosts as you achieve, with one exception:

You can only achieve an A+ in this class if you receive a base A grade.
The A+ grade does not have an impact on your GPA, and so I am reserving
it as a way of acknowledging especially excellent work across the course
of the semester.

\subsubsection{Test crusher}

Each additional E you earn above what was required for your base letter
grade will round your grade up (max 2 round-ups).

\subsubsection{Time Management}

If you use no more than 3 extension requests all semester, I will round
your grade up.

\subsubsection{Completionist}

If you get 100\% on the homework or the pre-class quizzes, I will round
your grade up (one round-up only).

\subsubsection{Final Project Ace}

If you do go above and beyond expectations on your final project, I will
round your grade up (this is at my discretion).

\subsubsection{Growth Boost}

If you demonstrate consistent improvement in your work across the
semester, I will round your grade up.

\section{Deadlines and Extensions}\label{deadlines}

If you need an extension, you may request one using
\href{https://forms.gle/eFx7y7FoSdoukKGC6}{this form}. I recognize that
things come up and you may require flexibility at some point in the
semester. Please feel free to reach out to me directly if you are
struggling to meet a deadline. I want to support you and make sure you
have the best possible chance for success in this class, and the only
way I can help is if you communicate with me. Extension requests are
always due before the deadline. Work submitted after the deadline
without an approved extension will not be graded.

In general, I am happy to be flexible. Please note, however, that some
assignments will have stricter deadlines. These assignments include the
reading quizzes and the final project presentation, and the nature of
the deadlines is discussed in their descriptions above.



\end{document}
